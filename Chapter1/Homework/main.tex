\documentclass{article}
\usepackage[utf8]{inputenc}

\title{Chapter 1 Homework}
\date{28 August 2023, due 6 September}

\begin{document}

\maketitle

You are tasked with modeling an outbreak of the common cold (\emph{Rhinovirus}) in a city. First, you can assume that population size $N$ is approximately fixed for the duration of the outbreak. It is known that for \emph{Rhinovirus}, individuals are either susceptible or infected, and that someone may be infected multiple times during a season. In other words, the progression of the disease is

\begin{equation} S \rightarrow I \rightarrow S \end{equation} 

\noindent where $S,I$ are the number of individuals in the susceptible and infected classes at time $t$ \textbf{(thus $\mathbf{S+I=N}$)}. The rate of transition from $S$ to $I$ is given by the probability of contacting a sick individual ($I/N$) multiplied by the transmission rate for a contact ($\beta : \beta > 0 $). Finally, individuals recover from a common cold at a rate $\gamma$ ($\gamma > 0$). 

\begin{itemize}
    \item Write down a difference equation that describes the dynamics of this system. (NB: Only one equation is needed.)
    \item What are the two fixed points for the system? What do the fixed points tell us about the conditions on $\beta$ and $\gamma$?
    \item Using recursion, vary the three parameters ($N,\beta,\gamma$) of the system and plot the results for each combination; choose 3 different values for each parameter. Include at least one combination where the condition in the previous item is broken. Assume that $I_0=1$. Do the plots confirm your analytical results?
\end{itemize}

Next, consider a slight modification on the biology of the common cold such that individuals spend sometime after infection where they a not susceptible to infection (i.e., temporary immunity). Now the progression of the disease is 

\begin{equation} S \rightarrow I \rightarrow R \rightarrow S \end{equation} 

\noindent where $R$ is the ``recovered'' class with temporary immunity. This time, let $\gamma$ be the transition rate from $I\rightarrow R$ and $\mu$ be the transition rate from $R \rightarrow S$.

\begin{itemize}
    \item Repeat the exercises for the SIS system for the SIRS system. This time, do not vary the parameter $N$; rather, vary the parameter $\mu: \mu > 0.$ (NB: you will now need a system of equations.)
\end{itemize}

\end{document}
